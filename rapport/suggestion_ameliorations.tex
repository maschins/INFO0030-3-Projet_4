% !TEX root = ./rapport.tex

\section{Suggestions d'améliorations}

Pendant le développement de notre Mastermind, nous avons envisagé plusieurs améliorations. Une idée principale était d'incorporer une ambiance sonore, comprenant une sélection de musiques de fond adaptées ainsi que des effets sonores pour accompagner les actions du joueur, comme des applaudissements pour une victoire ou des sons distincts lorsqu'on clique sur les différents boutons.
\\\\
De plus, nous avons envisagé d'offrir une personnalisation avancée de l'interface du jeu. Des réglages auraient été accessibles via une fenêtre de paramètres, permettant au joueur d'ajuster des éléments tels que la taille de la fenêtre de jeu, la couleur ou l'image de fond, ainsi que le volume de la musique et des effets sonores.
\\\\
Nous avons également pensé à l'ajout d'animations pour rendre le jeu plus dynamique et attrayant. Ces animations auraient inclus des transitions fluides entre le menu et le jeu, des effets visuels lors de la sélection des couleurs, et des animations spéciales pour marquer la découverte d'un pion bien positionné ou d'une combinaison correcte.
\\\\
De plus, nous avons envisagé l'implémentation de différents niveaux de difficulté pour offrir un défi adapté à tous les types de joueurs, allant des débutants en mode "proposer" préférant un feedback vérifié par l'ordinateur, aux joueurs expérimentés en mode "guesser" qui souhaitent limiter le nombre de combinaisons possibles.
\\\\
Enfin, nous aurions aimé créer un système de score plus réaliste qui fonctionnerait avec des bonus de vitesse de réponse ou encore des multiplicateurs de score liés à la difficulté du jeu. 