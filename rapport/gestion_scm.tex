% !TEX root = ./rapport.tex

\section{Gestion du SCM}

Au début de notre projet, l'utilisation de GitLab s'est avérée compliquée. Nous avons rencontré des difficultés lors de sa configuration et de sa prise en main, ce qui a considérablement ralenti le début du développement de notre projet. Une fois ces problèmes techniques résolus, nous avons constaté que les commits n'étaient pas suffisamment réguliers, ce qui a entraîné des difficultés à intégrer le travail de chacun. Nous avons également dû faire face à des conflits de fusion lors des pulls ce qui a entraîné des pertes de code.
\\\\
Cependant, à force d'utiliser et en comprenant davantage le fonctionnement de GitLab, nous avons commencé à travailler simultanément sur différentes parties du projet. Cela nous a permis de mieux organiser notre travail et de conserver des versions antérieures plus régulières du code. Au final, l'utilisation de GitLab nous a permis de rendre le processus de collaboration beaucoup plus fluide et rapide.
\\\\
Vous pouvez retrouver notre projet sur GitLab via le lien suivant :\\ \url{https://gitlab.uliege.be/Martin.Schins/info0030_groupe10}